Part of our work at the European project P2P-Next \cite{p2p-next} is P2P-Tube, a web platform which aims to be a complete solution for video sharing, video-on-demand and live streaming through Next-Share \cite{next-share} content delivery platform. Users are able to watch videos through one of the two Next-Share browser plugins, one based on HTML5 and the other based on VLC \cite{vlc}. They can sign up for an account using three different ways of authentication: internal P2P-Tube authentication, authentication through LDAP \cite{ldap} or through OpenID \cite{openid}, using a third-party account. Logged in users can share their thoughts by uploading new video assets. Social interaction between them is possible through the possibility of commenting each other's videos and voting them with likes and dislikes.

In order to make P2P-Tube scale to a large amount of users which are concurrently uploading new video assets, we have designed a distributed system which uses one or more Content Ingestion Servers (CIS). Their role is to prepare uploaded videos for sharing with other users and make those videos available on the platform. We have chosen web services as the way of communication between web servers, which deliver P2P-Tube application to the users, and Content Ingestion Servers.

Next-Share technology and plugins used in our platform are presented in \S \ref{sec:next-share}. P2P-Tube platform design architecture and implementation are presented in \S \ref{sec:design} and \S \ref{sec:implementation}, respectively. We propose future work in \S \ref{sec:future-work} and conclude this article in \S \ref{sec:conclusion}.
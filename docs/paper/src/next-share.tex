P2P-Next is an FP7 European project made up by a consortium of academic and industrial players which aim to build the next generation \textit{peer-to-peer (P2P)} content delivery platform. 21 partners from 12 different countries are involved here: BBC, Technische Universiteit Delft, Pioneer, Technical Research Center of Finland and also University Politehnica of Bucharest, were P2P-Tube platform was developed. P2P-Next takes as design principals the usage of open standards, open source development and future proof iterative integration.

In the last ten years two emerging trends sparkled the evolution of Internet: first of all more and more users share their content through social networks and P2P systems and secondly video sharing, video-on-demand and live streaming are increasingly gaining popularity. The current Internet infrastructure was not initially designed for simultaneous transmission of live events to millions of people and proposed technologies like multicasting do not seem to solve the problem. Television is no longer the main medium for audio and visual information content. Motivated by all these fact, P2P-Next started research and development of a new multimedia infrastructure base on both P2P technologies like BitTorrent and video streaming.

Next-Share is the name of the content delivery technology provided by P2P-Next which enables features such as on-demand video and live video for both computers and STBs (digital set-top-boxes) usable with TVs.

Next-Share platform implements its core functionality into NextShareCore which is written in Python. Video rendering in a PC browser using Next-Share can be done with two types of plugins: SwarmPlayer and NextSharePC.

A de facto standard today for playing videos in a browser is Adobe Flash Player. Because this is a proprietary technology and P2P-Next uses open standards other technologies needed to be explored. W3C (World Wide Web Consortium) currently works at a new version of HTML, HTML5, which supports, besides other revolutionary features, audio and video tags. Thus video files can be easily embedded into web pages, just like images, without any browser extensions or plugins. Although HTML5 is still a draft, most modern browsers already implement some of its features including video and audio tags. On this background, non-profit organizations and big corporations have engaged in a codec war. Some of the problems were whether to accept or not proprietary condecs, which codecs should standardly accepted etc.. For instance Microsoft promotes AVC / H.264, a proprietary video codec. Despite of its image quality and its good compression ratio, non-profit organization criticized it for being proprietary and closed standard. As a consequence Ogg containers with Vorbis audio codec and Theora video codec were included into HTML5. Google also stepped into this war and acquired On2 Technologies, for VP8, an open video compression format. They proposed WebM containers with Vorbis audio compression and VP8 video compression. Because Google's proposal is a free open standard it was accepted into HTML5 along with Ogg (Vorbis + Theora). Currently most modern browsers support video tags with Ogg and WebM containers.

P2P-Next developed \textbf{SwarmPlayer}, a Next-Share compliant browser plugin which is based on HTML5. It is currently supported in Windows in Mozilla Firefox as an extension and in Internet Explorer and there is also a Mac version.

Because only a few video formats are supported in HTML5 and older browsers do not support HTML5 video tags P2P-Next developed another variant of its Next-Share plugin, named \textbf{NextSharePC}. This one is based on VLC Player libraries to render videos and is able to play a lot of video formats, basically anything that is support by VLC Player. Written by the VideoLAN project, VLC Media Player is a free open source cross-platform multimedia player and framework. Its disadvantage is that it is currently supported only in Windows, for both Mozilla Firefox, as a plugin, and Internet Explorer.

A NextShare plugin consists of two major parts: the browser plugin / extension, running in the browser address space and the Next-Share Agent API which facilitates communication with the NextShareCore.